\documentclass[12pt,a4paper]{article}
\usepackage[T1]{fontenc}
\usepackage[latin1]{inputenc}
\usepackage[english]{babel}
\usepackage[swedish]{babel}
\usepackage{fullpage}
\usepackage{graphicx}

\begin{document}
\title{Text-Based Calculator in SML}
\author{
  Vågbratt, Tommy
  \and
  Loberg, Micael
	\and
  Jin, Wenting}
\date{\today}
\maketitle
\tableofcontents
\newpage


\section{Introduction}
\textnormal{A calculator that takes mathematical expressions in text form may sound nothing special, but one may have encountered difficulties to find the right symbol to express trigonometric related functions such as "sin","pi" etc, on a typing-in calculator. However, this calculator performs just as well as "normal" ones, but can evaluate a whole mathematical expression that is just based on text! No more time wasting looking up symbols. Interesting? Well, it's just a calculator.}

\section{Calculator in SML}
\subsection{Design and Structure}
\textnormal{}
\iffalse
En mer detaljerad beskrivning (design)
• Vilka delar består systemet av? Hur samverkar de för att lösa problemet?
• Vilka datastrukturer används? Beskriv abstrakta datatyper (gränssnitt/interface)
\fi

\subsection{Algorithms and Alternatives}
\textnormal{}
\iffalse
Smarta lösningar (algoritmer)
• Beskriv viktiga funktioner Designval
• Finns alternativa sätt att lösa problemet? Varför valde du det här sättet?
• Konsekvenser av dina val
\fi

\subsection{Implementation}
\textnormal{}
\iffalse
• Intressanta detaljer
• Fungerar programmet? Hur vet du det?
• Finns det saker som inte fungerar, fall som inte hanteras?
• Är programmet effektivt?
\fi

\subsection{General Analysis}
\textnormal{}
\iffalse
Analys/diskussion
• Vilka är systemets styrkor/svagheter? -> Sustainability
• Blev det bra? Skulle du ha gjort något annorlunda om du skulle börja om? -> Sustainability
• Tänkbara vidareutvecklingar -> Furture Development
\fi

\subsubsection{Sustainability}
\textnormal{}

\subsubsection{Furture Development}
\textnormal{}

\section{Brief Conclusion}
\textnormal{}
\iffalse
• Avslutning: Sammanfattning, diskussion, slutsatser
\fi

\section{Simple Guide for Simple Calculator}
\textnormal{}
\iffalse
• Användarhandledning med exempel (use cases)
\fi

\begin{thebibliography}{breitestes Label}
\bibitem{%cite tag}
  % info...
  \emph{% title}
	
\bibitem{} \url{%link}
	
\end{thebibliography}



\end{document}
